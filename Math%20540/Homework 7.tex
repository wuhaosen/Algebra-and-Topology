%\documentclass[11pt]{amsproc}
%\documentclass[11pt]{article}
\documentclass[11pt]{article}
%\usepackage{setspace}
%\usepackage{fancyhdr}
\usepackage{fullpage}
\usepackage{graphicx}
\usepackage{amssymb}
%\usepackage{accents}
\usepackage{amsfonts}
\usepackage{amsthm}
\usepackage{amsmath}
\usepackage{eucal}
\usepackage{xypic}
\usepackage{pdfsync}
\usepackage{hyperref}
\usepackage{enumerate}



%%\setrightmargin{1in}
%\setallmargins{1in}

% Titlerule is a FAT ruler
\newcommand{\titlerule}{\rule{\linewidth}{1.5mm}}
% For comments in the draft - work in progress
\newcommand{\betainsert}[2]{\fbox{#1}\marginnote{\textsf{#2}}}

% Notes in the margin are nicer this way. HaHa
\newcommand{\marginnote}[1]{\marginpar{\scriptsize\raggedright #1}}



\def\bd{\partial}
\def\ra{\rightarrow}
\def\lra{\longrightarrow}
\def\Z{{\mathbb Z}}
\def\N{{\mathbb N}}
\def\R{{\mathbb R}}
\def\Q{{\mathbb Q}}
\def\C{{\mathbb C}}
\def\P{{\mathbb P}}
\def\K{{\mathbb K}}
\def\w{\mathcal{W}(E)}
\def\A{\mathcal{A}}
\def\B{\mathcal{B}}
\def\M{\mathcal{M}}
\def\N{\mathcal{N}}
\def\p{\partial}

\newcommand*{\longhookrightarrow}{\ensuremath{\lhook\joinrel\relbar\joinrel\rightarrow}}

\newtheorem{lem}{Lemma}
\newtheorem{prop}{Proposition}
\newtheorem{thm}{Theorem}
\newtheorem{cor}{Corollary}
\newtheorem{conj}{Conjecture}
\newtheorem{defn}{Definition}
\newtheorem{claim}{Claim}
\newtheorem{ques}{Question}
\newtheorem{rem}{Remark}

\theoremstyle{remark}
\newtheorem*{prob}{Problem}
\newtheorem{ex}{Example}
\def\T{\mathbb{T}}

\begin{document}
\begin{center}
    \begin{Large} {\bf Math 540 Homework 7}\\
    \end{Large}
    Haosen Wu  / Thur, Spet. 27, 2018
\end{center}
%\vspace{10mm}

\subsection*{1}
    Indeed we are proving the diagram below commutes:
    \[
    \xymatrix{
    & \tilde{X}/F \ar[d]^{\theta}  \ar[r]^{\rho(-)} & \Tilde{X}/F' \ar[d]^{\theta} &  \\
    & \tilde{X'}/F  \ar[r]^{\rho'(-)} & \Tilde{X'}/F' &
    }
    \]
    (Psychologically we distiguish $F,F'$ but $F=F'$ in fact.)
    
    The isomorphism of covering $\varphi: \tilde{X}\rightarrow \tilde{X'}$, combine with path lifting property (asserts the uniqueness), tells us that the lifting of path $\alpha$ whose $[\alpha] \in \pi_1(X;x_0)$ induces identical monodromy antihomomorphism $\rho(-)$ and $\rho'(-)$. But then (letting $x_0\in F$) $\rho([\alpha])\circ \theta(x_0)$ is sending $x_0$ first to some $y_0\in F$, then to $z'_0\in F'$; $\theta \circ \rho ' ([\alpha])$ is sending $x_0$ first to some $w'_0\in F'$, then to $z'_0\in F'$, since $w'_0=\rho([\alpha])(x_0),z'_0=\rho([\alpha])(y_0)$ as well $\rho([\alpha])$ preserves the sending. Therefore $ \theta \circ \rho ' ([\alpha]) = \rho([\alpha])\circ \theta(x_0)$, the diagram commutes, which is equivalent to say $\rho ([\alpha])=\theta \circ \rho ' ([\alpha]) \circ \theta^{-1}$.
    
    (Abuse of notation here)  
    
    Conversely, if diagram commutes, map $\rho(-)$ can construct map from covering $\tilde{X}$ to covering $\tilde{X'}$ by gluing up fibers, as ball around fiber provides natural local homeomorphism; Inverse of $\rho(-)$ provides inverse map from covering $\tilde{X'}$ to covering $\tilde{X}$. Then the map between coverings provides an  isomorphism.
    
    (The converse part has been proved in class, as realizing the same $\rho(-)$ for both coverings. Some abstract proof can be given as: identify elements in Bij(F) as group action elements in $S_n$, $n$ is cardinality of fiber, then result is immediate from elements in $S_n$ conjugates iff they have same cycle type. This proof is good but only when fiber is finite.) 
    
\subsection*{General Lemma for 2 and 3}
  \indent (For Connected Covering) We investigate $p_*(\pi(\tilde{X};\tilde{x_0})\in \pi(X;x_0)$
  
  (For All Covering) We follow the fashion in Problem 1,  which tells us $n$-sheet covering space is \textit{really} classified by equivalence classes of homomorphisms from $\pi_1(X;x_0)$ to $S_n$. 
\subsection*{2 & }
    
    Not fixing F, group $\Z\times\Z$ has following homomorphisms to $S_3$: $|Image|=2$, $\Z/2\times\Z$ embeds in $\Z/2 \leq S_3$; Image=3, $\Z/3\times\Z$ embeds in $\Z/3 \leq S_3$. Thus we have two non-isomorphic (class of) coverings.  
    
    When fixing F, we talk about: monodromy action inducing homomorphism with $|Image|=1$, we have trivial bundle (with torus structure) covering, this is also indeed the case that fibre element get fixed since it is identity permutation, and we also know within the isomorphism class we could have $3!$ many coverings. 
    
\subsection*{ & 3}    
    
    When not fixing F, group $K=\langle a,b; a^2=b^2 \rangle$ has following homomorphisms to $S_3$: We need some analysis here, $|Image|=2$, we mapped group $K$ to subgroup  $\Z/2 \leq S_3$. This morphism is possible since we can set first $i_1(1\in \Z)=1\in \Z/2$ and second $i_2(1\in \Z)=1\in \Z/2$ the homomorphism $\varphi$ by juxtaposition. then $\varphi(\langle a,b; a^2=b^2 \rangle)=\langle a; a^2=0 \rangle$ which is precisely $\Z/2 \leq S_3$. Therefore we obtained one class of isomorphic covering. 
    
    \[
    \xymatrix{
    \Z \ar[r]^{}\ar[dr]^{i_1} & \langle a,b; a^2=b^2 \rangle \ar[d]^{\varphi}&  \Z \ar[l]_{} \ar[dl]_{i_2}\\
    & \Z/2 \subset S_3 &
    }
    \]
    
    In the same vein we map $i_1(1\in \Z)=1\in \Z/3$ and second $i_2(1\in \Z)=2\in \Z/3$, which is the only plausible map. Computation shows that such generated $\varphi$ will send first $\varphi(1\in \Z)=1 \in \Z/3$, second $\varphi(1\in \Z)=2 \in \Z/3$, but then first $\varphi((1\in \Z)^2)=2 \in \Z/3$ and second $\varphi((1\in \Z)^2)=(2 \in \Z/3)^2=1 \in \Z/3$. This contradicts with well-defindness of $\varphi$ and relation $a^2=b^2$, therefore, there is no correspond covering.
    
    When fixing F, we talk about: $|Image|=1$, we have the trivial bundle (with Klein-bottle structure) covering; this is also the case that fibre elements get fixed since it is identity permutation, and we also know within the isomorphism class we could have $3!$ many coverings (as well in 2).  
    
\subsection*{ & }     
    Thus we completed 3-sheets covering classification for 2 & 3.
\subsection*{4}
    lift $\rho{[\alpha]}(\tilde{x_0})=$NumbNumbNumb. Nice parking lot! 

\end{document}
\frac{\pi}{}