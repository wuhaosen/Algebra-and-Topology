%\documentclass[11pt]{amsproc}
%\documentclass[11pt]{article}
\documentclass[11pt]{article}
%\usepackage{setspace}
%\usepackage{fancyhdr}
\usepackage{fullpage}
\usepackage{graphicx}
\usepackage{amssymb}
%\usepackage{accents}
\usepackage{amsfonts}
\usepackage{amsthm}
\usepackage{amsmath}
\usepackage{eucal}
\usepackage{xypic}
\usepackage{pdfsync}
\usepackage{hyperref}
\usepackage{enumerate}



%%\setrightmargin{1in}
%\setallmargins{1in}

% Titlerule is a FAT ruler
\newcommand{\titlerule}{\rule{\linewidth}{1.5mm}}
% For comments in the draft - work in progress
\newcommand{\betainsert}[2]{\fbox{#1}\marginnote{\textsf{#2}}}

% Notes in the margin are nicer this way. HaHa
\newcommand{\marginnote}[1]{\marginpar{\scriptsize\raggedright #1}}



\def\bd{\partial}
\def\ra{\rightarrow}
\def\lra{\longrightarrow}
\def\Z{{\mathbb Z}}
\def\N{{\mathbb N}}
\def\R{{\mathbb R}}
\def\Q{{\mathbb Q}}
\def\C{{\mathbb C}}
\def\P{{\mathbb P}}
\def\K{{\mathbb K}}
\def\w{\mathcal{W}(E)}
\def\A{\mathcal{A}}
\def\B{\mathcal{B}}
\def\M{\mathcal{M}}
\def\N{\mathcal{N}}
\def\p{\partial}

\newcommand*{\longhookrightarrow}{\ensuremath{\lhook\joinrel\relbar\joinrel\rightarrow}}

\newtheorem{lem}{Lemma}
\newtheorem{prop}{Proposition}
\newtheorem{thm}{Theorem}
\newtheorem{cor}{Corollary}
\newtheorem{conj}{Conjecture}
\newtheorem{defn}{Definition}
\newtheorem{claim}{Claim}
\newtheorem{ques}{Question}
\newtheorem{rem}{Remark}

\theoremstyle{remark}
\newtheorem*{prob}{Problem}
\newtheorem{ex}{Example}
\def\T{\mathbb{T}}

\begin{document}
\begin{center}
    \begin{Large} {\bf Math 540 Homework 1}\\
    \end{Large}
    Haosen Wu  / Friday, August 29, 2018
\end{center}
%\vspace{10mm}


\subsection*{1 Concrete example of homotopy}
 Fill the boxes below there.
 
 We need to verify that for any $a \in \pi_1(X;x_0) $ the pushouts $f_*(a) = T_\gamma g_*(a) \in \pi_1(Y;y_0) $. Now we have the homotopy G from a), which tells us $f\circ a \simeq \gamma * (g \circ a) * \gamma^{-1}$, w.r.t. $\pi_1(Y;z_0)$. Thus the last equality holds.


\subsection*{2 Path homotopy equivalent to same path connected component}
  By definition of $h(t)$ the begin/end point satisfies, we need to show $h(t)$ is continuous. For any $\epsilon >0$, take $d(t_1,t_2)< \delta$, then $d_1(h(t_1),h(t_2)) = d_1(h_{t_1},h_{t_2}) = \textrm{sup } d_0(h_{t_1}(s),h_{t_2}(s))  = \textrm{sup } d_0(H(t_1,s),H(t_2,s)) <\epsilon$, since H is a path homotopy.

  Again by definition of $H(s,t)$ this map connects the begin/end point $\alpha,\beta$, we need to show $H(s,t)$ is continuous. For any $\epsilon >0$, take $d((t_1,s_1),(t_2,s_2))< \delta$, then $d_0(H(t_1,s_1),H(t_2,s_2))\leq d_0(H(t_1,s_1),H(t_1,s_2))+d_0(H(t_1,s_2),H(t_2,s_2)) = d_0(h_{t_1}(s_1),h_{t_1}(s_2)) + d_1(h_{t_1},h_{t_2}) < \epsilon + \epsilon = 2\epsilon$. The first estimate follows from $h_{t_1}$ is a path in $X$ and the second from $h$ is a path in $\Omega_{x_0 y_0}X$.

  Thus combine these two directions give the proof.
\subsection*{3 Uniqueness of homotopy equivalence}

\begin{lem}
  If maps $f_1\simeq f_2$ then maps $g\circ f_1 \simeq g\circ f_2$ 
  \begin{proof}
  Trivial
  \end{proof}
\end{lem}
  We know now, by the given and lemma, that $h\circ f\circ k \simeq h\circ Id_Y \simeq Id_X \circ k$, we claim this map $h\circ f\circ k$ is $g$. Note $f\circ g \simeq f\circ (Id_X\circ k) \simeq f\circ Id_X\circ k \simeq f\circ k \simeq Id_Y$, similarly $g\circ f \simeq Id_X$ and we are done.

\end{document}
