%\documentclass[11pt]{amsproc}
%\documentclass[11pt]{article}
\documentclass[11pt]{article}
%\usepackage{setspace}
%\usepackage{fancyhdr}
\usepackage{fullpage}
\usepackage{graphicx}
\usepackage{amssymb}
%\usepackage{accents}
\usepackage{amsfonts}
\usepackage{amsthm}
\usepackage{amsmath}
\usepackage{eucal}
\usepackage{xypic}
\usepackage{pdfsync}
\usepackage{hyperref}
\usepackage{enumerate}



%%\setrightmargin{1in}
%\setallmargins{1in}

% Titlerule is a FAT ruler
\newcommand{\titlerule}{\rule{\linewidth}{1.5mm}}
% For comments in the draft - work in progress
\newcommand{\betainsert}[2]{\fbox{#1}\marginnote{\textsf{#2}}}

% Notes in the margin are nicer this way. HaHa
\newcommand{\marginnote}[1]{\marginpar{\scriptsize\raggedright #1}}



\def\bd{\partial}
\def\ra{\rightarrow}
\def\lra{\longrightarrow}
\def\Z{{\mathbb Z}}
\def\N{{\mathbb N}}
\def\R{{\mathbb R}}
\def\Q{{\mathbb Q}}
\def\C{{\mathbb C}}
\def\P{{\mathbb P}}
\def\K{{\mathbb K}}
\def\w{\mathcal{W}(E)}
\def\A{\mathcal{A}}
\def\B{\mathcal{B}}
\def\M{\mathcal{M}}
\def\N{\mathcal{N}}
\def\p{\partial}

\newcommand*{\longhookrightarrow}{\ensuremath{\lhook\joinrel\relbar\joinrel\rightarrow}}

\newtheorem{lem}{Lemma}
\newtheorem{prop}{Proposition}
\newtheorem{thm}{Theorem}
\newtheorem{cor}{Corollary}
\newtheorem{conj}{Conjecture}
\newtheorem{defn}{Definition}
\newtheorem{claim}{Claim}
\newtheorem{ques}{Question}
\newtheorem{rem}{Remark}

\theoremstyle{remark}
\newtheorem*{prob}{Problem}
\newtheorem{ex}{Example}
\def\T{\mathbb{T}}

\begin{document}
\begin{center}
    \begin{Large} {\bf Math 540 Homework 6}\\
    \end{Large}
    Haosen Wu  / Thur, Spet. 27, 2018
\end{center}
%\vspace{10mm}

\subsection*{1}
Consider lifting 
    \[
    \xymatrix{
    & & \Tilde{X} \ar[d]^{p}&  \\
    & Y \ar[r]^{f}\ar[ur]^{\Tilde{f}} & X
    }
    \]
The fundamental group of Y being finite suggests that, since $\pi_1(X)=\Z$, the pushout $f_*$ (as homomorphism) has to be trivial, thus we have desired lifting $\tilde{f}$. But $\tilde{X}=\R$ suggests that $\tilde{f}$ is null-homotopic, thus $f=p\circ \tilde{f}$ is null-homotopic as well. 

\subsection*{2}
Let $U$ be an open ball in $X$ and $i$ be the inclusion map
    \[
    \xymatrix{
    & & \Tilde{X} \ar[d]^{p}&  \\
    & U \ar[r]^{i}\ar[ur]^{\Tilde{i}} & X
    }
    \]
Now we have the universal cover of $X$, consider lifting $\tilde{i}$ in $\Tilde{X}$, it always exists since $i$ is merely inclusion. Suppose we do not have semilocally simply connectedness, then we have a contradiction, since this is equivalent to say $i_*(\pi_1(U)) \not\subset p_*(\pi_1(\tilde{X}))$, therefore $X$ has to be semilocally simply connected.

\subsection*{3}
Consider family of ball $\{B_r(0)\}_r$, each $B_r(0)$ contains a loop, the map $\pi_1(B_r(0),0)\rightarrow\pi_1(X,0)$ is non-trivial (guaranteed by map $r_n$ in homework 4.3.a),  which translates to non-semilocally simply connectedness.    

\subsection*{4}
\begin{enumerate}
    \item Let $f(\tilde{x})=[p\circ \tilde{\alpha}]$, path class independent of choice of elements in $p_*(\pi_1(\tilde{X})) $, as if we pick $\alpha=p\circ \tilde{\alpha} \in [p\circ \tilde{\alpha}]$, the concatenation with $\beta$ which $\beta \in p_*(\pi_1(\tilde{X})) $, $\beta*\alpha$ has its lift $\tilde{\beta}*\tilde{\alpha}$ a fixed end since $\tilde{\beta}$ is necessarily a loop (A theorem in class). This proves the well-definedness. 
    \item $f$ is surjective since $\tilde{X}$ is path-connected thus we can always find the path connecting $\tilde{x_0}$ and $p^{-1}(x_0)$.
    \item Suppose $f(\tilde{y})=[p\circ \tilde{\beta}]$, once representatives $p\circ \tilde{\alpha} \neq p\circ \tilde{\beta}$, that is, $[\alpha*\beta^{-1}] \not\in p_*(\pi_1(\tilde{X}))$, $\alpha*\beta^{-1}$ is thus not  a loop. We can conclude that $\tilde{x}\neq \tilde{y}$. 
\end{enumerate}
Now the bijectivity proves the claim of correspondence of fiber counts - image index.

\end{document}

\frac{\pi}{}