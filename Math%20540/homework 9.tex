
%\documentclass[11pt]{amsproc}
%\documentclass[11pt]{article}
\documentclass[11pt]{article}
%\usepackage{setspace}
%\usepackage{fancyhdr}
\usepackage{fullpage}
\usepackage{graphicx}
\usepackage{amssymb}
%\usepackage{accents}
\usepackage{amsfonts}
\usepackage{amsthm}
\usepackage{amsmath}
\usepackage{eucal}
\usepackage{xypic}
\usepackage{pdfsync}
\usepackage{hyperref}
\usepackage{enumerate}
\usepackage{tikz-cd}



%%\setrightmargin{1in}
%\setallmargins{1in}

% Titlerule is a FAT ruler
\newcommand{\titlerule}{\rule{\linewidth}{1.5mm}}
% For comments in the draft - work in progress
\newcommand{\betainsert}[2]{\fbox{#1}\marginnote{\textsf{#2}}}

% Notes in the margin are nicer this way. HaHa
\newcommand{\marginnote}[1]{\marginpar{\scriptsize\raggedright #1}}



\def\bd{\partial}
\def\ra{\rightarrow}
\def\lra{\longrightarrow}
\def\Z{{\mathbb Z}}
\def\N{{\mathbb N}}
\def\R{{\mathbb R}}
\def\Q{{\mathbb Q}}
\def\C{{\mathbb C}}
\def\P{{\mathbb P}}
\def\K{{\mathbb K}}
\def\w{\mathcal{W}(E)}
\def\A{\mathcal{A}}
\def\B{\mathcal{B}}
\def\M{\mathcal{M}}
\def\N{\mathcal{N}}
\def\p{\partial}

\newcommand*{\longhookrightarrow}{\ensuremath{\lhook\joinrel\relbar\joinrel\rightarrow}}

\newtheorem{lem}{Lemma}
\newtheorem{prop}{Proposition}
\newtheorem{thm}{Theorem}
\newtheorem{cor}{Corollary}
\newtheorem{conj}{Conjecture}
\newtheorem{defn}{Definition}
\newtheorem{claim}{Claim}
\newtheorem{ques}{Question}
\newtheorem{rem}{Remark}

\theoremstyle{remark}
\newtheorem*{prob}{Problem}
\newtheorem{ex}{Example}
\def\T{\mathbb{T}}

\begin{document}
\begin{center}
    \begin{Large} {\bf Math 540 Homework 9, 10?}\\
    \end{Large}
    Haosen Wu  / Wed, Nov.14 , 2018
\end{center}
%\vspace{10mm}


\subsection*{1}
Some computation and as Hatcher gives: $H$ == commutative diagrams.
\begin{itemize}
\end{itemize}


\subsection*{2}
We know the deformation retraction of $A$ and $X$ are corresponding to isomorphic homology modules of $H_{n}(A)=H_{n}(X)$, therefore we adopt the (long) exact sequence given as:

\begin{tikzcd}
  \cdots \rar   & H_{n+1}(A) \rar & H_{n+1}(X) \rar & H_{n+1}(X,A)
          \ar[out=0, in=180, looseness=2, overlay]{dll}   & \\
        & H_{n}(A) \rar & H_{n}(X) \rar & H_{n}(X,A)
          \ar[out=0, in=180, looseness=2, overlay]{dll}   & \\
        & H_{n-1}(A)  \rar & H_{n-1}(X) M \rar & H_{n-1}(X,A) \rar & ...
\end{tikzcd} 

Now we find $H_{n}(X,A)$: Name $H_{n+1}(A)=H_{n+1}(X)=M$ if dimension is one down then $M^{(1)}$, we have some information: $$ker(H_{n+1}(A)\rightarrow H_{n+1}(X))=Im(0\rightarrow H_{n+1}(A))=0 $$ and $$M/\{0\}=Im(H_{n+1}(A)\rightarrow H_{n+1}(X))=ker(H_{n+1}(X)\rightarrow H_{n+1}(X,A))=M $$

Apply Isomorphism theorem, $$Im(H_{n+1}(A)\rightarrow H_{n+1}(X,A))=ker(H_{n+1}(X,A)\rightarrow H_{n}(A))=\{0\}$$
Thus $$H_{n+1}(X,A)/\{0\}=Im(H_{n+1}(X,A)\rightarrow H_{n}(A))=ker(H_{n}(A)\rightarrow H_{n}(X))=\{0\}$$
Eventually we have $$H_{n+1}(X,A)=\{0\}$$.

As our $n$ is arbitrarily chosen, the claim follows. 


\subsection*{3 Fun}
\begin{itemize}
    \item [a)] map $i_{Y\rightarrow X}$ defines a natural inclusion of chain module $i_{Y\rightarrow X}^*: C_n(Y)\rightarrow C_n(X)$, then they descends to $$\Tilde{i}_{Y\rightarrow X}^*: C_n(Y)/C_n(Z)\rightarrow C_n(X)/C_n(Z)$$; similarly we have $$\Tilde{p}_{Z\rightarrow X}^*: C_n(X)/C_n(Z)\rightarrow C_n(X)/C_n(Y)$$. 
    
    Now we need to show that $Im(\tilde{i})=Ker(\tilde{p})$ (though topologically trivial: $Y-Z=(X-Z)-(X-Y)$). We notice that $Im(\tilde{i})=C_n(Y)/C_n(Z)$, and simplices are in linear space (then we just need to show below holds for one simplex).
    we show $Im(\tilde{i})\subset Ker(\tilde{p})$: pick an element in $[c] \in Im(\tilde{i})$, we have a representative $c=c_Y+c_Z$ where $[c_Z]=0$. Since $X\supset Y$ we also have $C_n(X)/C_n(Z)\supset C_n(Y)/C_n(Z)$, now $$\tilde{p}[c]=\tilde{p}(c_Y+c_Z)=0$$ since $\tilde{p}$ nullifies $C_n(Y)$. \\
    
    We show $Im(\tilde{i})\supset Ker(\tilde{p})$: (harder...constructing...)  \\
    Thus after proof of two side inclusion we are done. 
    
    \item[b)] We just consider putting the short exact sequences from a) together and define $f:=\{f_n\}_n$, $g:=\{g_n\}_n$. We only need to show each square in the diagram commutes, then the exactness of chain complexes follows from exactness of each rows. Now the squares commutes since $\tilde{\p}$ still a boundary map, and $\Tilde{i}_{Y\rightarrow X}^*$ is indeed $Id_{Y\rightarrow X}^*:C_n(Y)/C_n(Z)\rightarrow C_n(X|Y)/C_n(Z)$. Then the commutativity follows from original diagram (...Hatcher?)XD
    
    \item[c)] We explicitly expand the short exact sequences from b) together then apply snake lemma to obtain the long exact sequence.
\end{itemize}


\subsection*{4 Fun}
\begin{itemize}
    \item [a)] We know the deformation retraction of $\R^m-\{x_0\}$ and $\R^n-\{x_0\}$ are corresponding to $S^{m-1}$ and $S^{n-1}$, but the homology modules of $S^{m-1}$ and $S^{n-1}$ are known to be different. If such homeomorphism exists then homology modules as homotopy invariant will be congruent. This reveals a contradiction.
    
    \item [b)] Continue the fashion of contradiction, suppose such homeomorphism exists then homology modules $H_{n}(R^m)=H_{n}(R^m)$, therefore also holds for their relative homology  $H_{n}(R^m,\{x_0\})=H_{n}(R^n,\{y_0\})$, since the point itself is a deformation retraction, we thus have $$H_{n}(R^m,\{x_0\})=H_{n}(R^m-\{x_0\},\{x_0\}-\{x_0\})=H_{n}(R^n-\{y_0\},\{y_0\}-\{y_0\})=H_{n}(R^n,\{y_0\})$$ 
    We notice, therefore $H_{n}(R^m-\{x_0\})=H_{n}(R^n-\{y_0\})$, but this contradicts with the conclusion in the previous statement.
\end{itemize}


\subsection*{5}
\begin{itemize}
    \item [a)] the map $\tau_n$ is a homeomorphism thus induce isomorphisms in homology module (chains), therefore the diagrams commutes.
    \item[b)] Use the commutativity and notice that $H_{n}(S^0)\rightarrow H_{n}(S^0)$ is literally flipping sign of two points, therefore the composition ...
    \item[c)] Use results from b) we know the $n+1$-th $H_n(\tau_n)$ are all deduced from $n$-th maps, thus after composition...
    \item[d)] you know ... 
\end{itemize}


\end{document}