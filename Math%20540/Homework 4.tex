%\documentclass[11pt]{amsproc}
%\documentclass[11pt]{article}
\documentclass[11pt]{article}
%\usepackage{setspace}
%\usepackage{fancyhdr}
\usepackage{fullpage}
\usepackage{graphicx}
\usepackage{amssymb}
%\usepackage{accents}
\usepackage{amsfonts}
\usepackage{amsthm}
\usepackage{amsmath}
\usepackage{eucal}
\usepackage{xypic}
\usepackage{pdfsync}
\usepackage{hyperref}
\usepackage{enumerate}



%%\setrightmargin{1in}
%\setallmargins{1in}

% Titlerule is a FAT ruler
\newcommand{\titlerule}{\rule{\linewidth}{1.5mm}}
% For comments in the draft - work in progress
\newcommand{\betainsert}[2]{\fbox{#1}\marginnote{\textsf{#2}}}

% Notes in the margin are nicer this way. HaHa
\newcommand{\marginnote}[1]{\marginpar{\scriptsize\raggedright #1}}



\def\bd{\partial}
\def\ra{\rightarrow}
\def\lra{\longrightarrow}
\def\Z{{\mathbb Z}}
\def\N{{\mathbb N}}
\def\R{{\mathbb R}}
\def\Q{{\mathbb Q}}
\def\C{{\mathbb C}}
\def\P{{\mathbb P}}
\def\K{{\mathbb K}}
\def\w{\mathcal{W}(E)}
\def\A{\mathcal{A}}
\def\B{\mathcal{B}}
\def\M{\mathcal{M}}
\def\N{\mathcal{N}}
\def\p{\partial}

\newcommand*{\longhookrightarrow}{\ensuremath{\lhook\joinrel\relbar\joinrel\rightarrow}}

\newtheorem{lem}{Lemma}
\newtheorem{prop}{Proposition}
\newtheorem{thm}{Theorem}
\newtheorem{cor}{Corollary}
\newtheorem{conj}{Conjecture}
\newtheorem{defn}{Definition}
\newtheorem{claim}{Claim}
\newtheorem{ques}{Question}
\newtheorem{rem}{Remark}

\theoremstyle{remark}
\newtheorem*{prob}{Problem}
\newtheorem{ex}{Example}
\def\T{\mathbb{T}}

\begin{document}
\begin{center}
    \begin{Large} {\bf Math 540 Homework 4}\\
    \end{Large}
    Haosen Wu  / Wednesday, Spet. 19, 2018
\end{center}
%\vspace{10mm}

\subsection*{Fundamental Group I}
  Recall SvK Theorem I. We have deformation retract diagrams as Figure $1$ for $\R^1,\R^2,\R^n$ for $3\leq n$ cases, labeled as $X_1,X_2,X_n$. $\R^1$ case is like line with $p$-divides, $\R^2$ case is like $B^2$ with $p$-holes, $R^n$ for $3\leq n$ deserves an explanation (intuitively we can "move points away"): 
    We reduce to that, for $\{U_i\} \in X_n$ be an open covering with following properties:\\
    \indent a) There exists a point $x_0$ such that $x_0$ ϵ $\{U_i\}$ for all $i$.\\
    \indent b) Each $\{U_i\}$ is simply connected.\\
    \indent c) If $i\neq j$, then $U_i \cap U_j$ is path-connected.\\
    Therefore our $X_n$ is simply connected. 
  (Another way is $R^n$  minus $p$ points is homotopy equivalent to a wedge sum of $p$ copies of $S^{n−1}$, c.f. Figure $1$). Thus correspondingly, for $X_1$ since the only path-connected space is line, $\pi_1(X_1)=1$;  for $X_2$ by diagram we proceed induction on $p$: for solely $1$ p we have $\pi_1({X_2}_1)=\pi_1({X_2}_2)=\Z$ and $\pi_1({X_2}_1\cap {X_2}_2)=1$ then we see all maps $i:\pi_1({X_2}_1\cap {X_2}_2) \rightarrow \pi_1({X_2}_i)$ are trivial. The amalgamated product degenerates to free product. Thus $\pi_1({X_2}_1\cup {X_2}_2)=\Z*\Z$. for $n$ p we have assume $\pi_1(\bigcup {X_2}_i)=*_n\Z$, we also recognize  $\pi_1((\bigcup_n {X_2}_i)\cap {X_2}_{n+1})=1$, and amalgamated product degenerates again thus $\pi_1((\bigcup_{n+1} {X_2}_i)=*_{n+1}\Z$; for $X_n$ since the space is simply and path connected we get $\pi_1(X_{2})=1$ immediately. 
  
\subsection*{Fundamental Group II}
  Recall SvK Theorem II.\\
  Note we have deformation retract from $X_1$ to Figure-$8$ (as Figure $2$) thus we have $\pi_1(X_1)=\Z*\Z$, $\pi_1(X_2)=1$, more importantly, $X_1\cap X_2$ is path connected, since it is the gluing part; we can also find deformation retract of $X_1$ to $X_1\cap X_2$ and $X_2$ to $X_1\cap X_2$.  (cf. Deformation retract in Figure $2$). Thus the colimit is $\pi_1(X_1)*_{\pi_1(X_1\cap X_2)}\pi_1(X_2)=\Z*_{\pi_1(X_1\cap X_2)}\Z$. 
  
  Now we proceed to find the relations of generators. We have orientation on $X_1\cap X_2\cong S^1$ as given, name the generator in $\Z$ as $[a]$. Then thus $i_1:\pi_1(X_1\cap X_2) \rightarrow \pi_1(X_1) $ sends $[a]\in \Z \textrm{ to } [ab] \in \Z*\Z$ ; and another map $i_2:\pi_1(X_1\cap X_2) \rightarrow \pi_1(X_2) $ sends $[a]\in \Z \textrm{ to } 1$ (Figure $2$).  So eventually $\Z*_{\pi_1(X_1\cap X_2)}\Z \cong <a,b; ab=1>$. \\�� 
  
  Since the self-intersection homotopy connects with $\partial X^1$
  The manifold is connected sum of two "almost $\R P^2$", a self-intersecting $8$-tube (Figure $2$).
  
\subsection*{Hawaiian Ring} 
    a. Since $X=\cup_n X_n$ we define $r_n$ as given map, the continuity on domain $X_n$ is clear since it is identity, on domain $\neg X_n$ is a constant function$(=0)$; finally on domain $X_n \cap \neg X_n=\{pt\}$ is $0$. Note $X_n \cap \neg X_n=\{pt\}$ is closed thus we can apply pasting lemma here.  \\
    b. To verify that $\gamma_\epsilon(t)$ is continuous we proceed mainly need to verify it is continuous at $(0,0)$ since when $s\neq 1/n (resp. \frac{1}{n+1})$ the map $\gamma_\epsilon(t)$ is continuous, due to composition of continuous functions.  When $s\rightarrow 1/n (resp. \frac{1}{n+1})$ we have $\gamma_\epsilon(-)\rightarrow (0,0)$, so the limit exists at $s=\frac{1}{n} (resp. \frac{1}{n+1})$ and pasting lemma asserts that our $\gamma_\epsilon(t)$ is continuous everywhere. \\
    c. Suppose we have $[\gamma_\epsilon] = [\gamma_\epsilon']$ that is $[\gamma_\epsilon] \simeq [\gamma_\epsilon']$ . Consider the pushforward  of continuous map $r_n$, we should have $r_n\circ \gamma_\epsilon \simeq r_n\circ \gamma_\epsilon' $ (by Homework 1.3). This composed path is totally in $X_n$, which is indeed a circle. By theorem in class we know there exists a isomorphism  $p:\pi(X_n;x_0) \rightarrow \Z$, then we expect to have $p(r_n\circ \gamma_\epsilon) = p(r_n\circ \gamma_\epsilon')$. However since we have different $\epsilon \textrm{ and } \epsilon'$, at least one elements in sequences are different, so $p([r_n\circ \gamma_\epsilon])=p(\dots * \Gamma^+ * \Gamma^- * \Gamma^+ *\dots) \neq p(\dots * \Gamma^+ * \Gamma^+ * \Gamma^+ *\dots) = p([r_n\circ \gamma_\epsilon'])$ at least the sign flipped from $-1$ to $1$ as $\Gamma$ flipped, thus we have different image in $\Z$. We arrived at contradiction so  $[\gamma_\epsilon] \neq [\gamma_\epsilon']$ \\ 
    d. Therefore different sequences $\epsilon, \epsilon'$ produces different homotopy classes, however we have uncountably many $\epsilon, \epsilon'$ that ranges in $\textrm{\{}-1,+1 \textrm{\}} $, therefore we have uncountably many elements in  $\pi_1(X)$. 
\end{document}